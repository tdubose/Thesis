\begin{enumerate}
\item Create or obtain phosphate buffered Milli-Q (PBM-Q), at least 500 mL.

\begin{enumerate}
\item Based on the work of Ellis and Hoidn \autoref{hoidnellis}, the salt concentration of the solution should be 10 mM with a pH of 7.6.

\item To make 500 mL PBM-Q, combine 130 mg $\mathrm{NaH_2PO_4\cdot H_2O}$, 1090 mg $\mathrm{Na_2HPO_4\cdot7H_2O}$, and 500 mL Milli-Q in a 500 mL sealable bottle. The Milli-Q should be measured out in a graduated cylinder.

\end{enumerate}

\item Next, bind the OPSS-PEG-NHS (OPN) to the antibody (Ab). We want approximately a 2:1 ratio of OPN:Ab. The number concentration of the Ab is \[\frac{8.92\times10^{15}\#\mathrm{Ab}}{mL}=\left(2\mathrm{\frac{mg}{mL}}\right)\left(\frac{6.02\times10^{23}\,\#\mathrm{Ab}}{135 kg}\right).\]
Measuring a quantity less than 1 mg of OPN is quite difficult, and we want to use a minimum pipette volume of 5 $\mu$L. Therefore, we will use 6 mg of OPN in 200 mL of PBM-Q. This gives a solution with
\[\frac{8.6\times10^{15}\#\mathrm{OPN}}{mL}=\left(6/200\mathrm{\frac{mg}{mL}}\right)\left(\frac{6.02\times10^{23}\,\#\mathrm{Ab}}{2.1 kg}\right).\]
Then, we combine 62.5 $\mu$L PBM-Q, 12.5 $\mu$L Ab, and 25 $\mu$L OPN+PBM-Q solution. Since the concentration of the Ab and OPN are approximately the same, and the volume used of the OPN+PBM-Q solution is double that of the Ab, there should be approximately a 2:1 OPN:Ab ratio. The actual procedure should be as follows:

\begin{enumerate}
\item Fill insulated container with ice.

\item In a 1.5 mL centrifuge tube, combine 62.5 $\mu$L PBM-Q and 12.5 $\mu$L Ab. Put tube in ice.

\item Measure out 200 mL PBM-Q and put in a sealable 250 mL bottle.

\item Measure out 6 mg OPN and add it to the 200 mL PBM-Q.

\item Quickly add 25 $\mu$L OPN+PBM-Q solution to the centrifuge tube, to minimize hydrolysis of the NHS ester.

\item Wait 24 hours for complete binding.

\end{enumerate}

\item Next, bind the OPAb to the Au nanospheres. We want 2000 OPAb\slash Au, and we want 500 mL of labeled spheres. The OPAb concentration is $1.1\times10^{15} \mathrm{\frac{\#OPAb}{mL}}$; therefore, we need
\[\left(500 \mu\mathrm{L} Au\right)\left(5\times10^9\mathrm{\frac{Au}{mL}}\right)\left(2000\mathrm{\frac{OPAb}{Au}}\right)/\left(1.1\times10^{15} \mathrm{\frac{\#OPAb}{mL}}\right)=4.5\mu\mathrm{L\ OPAb}.\] The procedure for this step is:

\begin{enumerate}
\item pH-correct\slash buffer Au solution:

\begin{enumerate}
\item Centrifuge 500 $\mu$L Au at 4200 RPM with soft brake for 10 minutes in a 5417C centrifuge.

\item Remove 2\slash 3 of the supernatant (333 $\mu$L) and replace with PBM-Q, making sure to resuspend the Au solution very well.

\end{enumerate}

\item Add 4.5 $\mu$L OPAb solution made the previous day to the Au solution.

\item Immediately DLS 50 $\mu$L of the OPAb-Au solution.

\item Wait 24 hours for OPAb-Au reaction to complete; again, DLS 50 $\mu$L.

\end{enumerate}

\item The final step of nanoparticle assembly is to add 5 kDa PEG-SH to the OPAb-Au. We want 10000 PEG-SH\slash Au for the 400 mL of remaining OPAb-Au solution.

\begin{enumerate}
\item First, make a PEG-SH+PBM-Q solution. 300 mL PBM-Q and 10 mg 5 kDa PEG-SH gives \[\frac{4\times10^{15}\#\mathrm{PS}}{mL}=\left(10/300\mathrm{\frac{mg}{mL}}\right)\left(\frac{6.02\times10^{23}\,\#\mathrm{Ab}}{5 kg}\right).\]

\item To get 10,000 PS\slash Au, we add
\[\left(400 \mu\mathrm{L} OPAb-Au\right)\left(5\times10^9\mathrm{\frac{OPAb-Au}{mL}}\right)\left(10000\mathrm{\frac{OPAb}{Au}}\right)/\left(4\times10^{15} \mathrm{\frac{\#PS}{mL}}\right)=5\mu\mathrm{L\ PS/}\]
to the 400 $\mu$L OPAb-solution.

\item DLS 50 $\mu$L OPAb-Au-PS immediately.

\item Wait 24 hours; DLS 50 $\mu$L OPAb-Au-PS.

\end{enumerate}

\item A final, optional step is to check the protection of the fully labeled and protected nanospheres. To do this, obtain a new centrifuge tube and put at least equal volumes (at least 50 $\mu$L) of both OPAb-Au-PS solution and commercial PBS solution. DLS 50 $\mu$L. Radius should be stable if the spheres are protected.

\end{enumerate}
