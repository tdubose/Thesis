Although the labeling experiments did not produced the hoped-for results of cellularly specific labeling, much progress has been made since the work of Perry Ellis and Oliver Hoidn in the summer of 2010. The PEGylation protocol has been completely optimized, and the spheres seem to be completely protected. In addition, it has been established that permeabilization permits Au nanoparticles to enter the cells, leading to an increase in nonspecific binding.

There are also a number of new and unanswered questions. Most obvious of these is determining why the 2Au preparation experiences greater binding than the 1--2Au preparation. There are two factors that determine the binding: the van der Waals binding, and the specific binding of the secondary antibody to the primary antibody. The van der Waals binding can be calculated numerically if the model of the labeled spheres is drawn up; even without performing the calculations, it is obvious that the 35 nm diameter nanospheres used by the originators of this project would generate far smaller van der Waals forces. Consequently, use of smaller nanospheres may be necessary to reduce the van der Waals binding. The specific binding, to date, has only been measured by the addition of 2Au to primary-labeled cells. However, there is an alternative method by which this could be measured. If OPAb-Au-PS spheres were to be made using the primary antibody (OPAb1-Au-PS) instead of the secondary (OPAb2-Au-PS), then a DLS measurement performed on a mixture of OPAb1-Au-PS and OPAb2-Au-PS should show the spheres agglomerating and precipitating out of solution as the primary- and secondary-labeled spheres form a lattice of sorts. The rate at which this occurs and the final state of the solution should give an approximate measure of the binding strength between the secondary and primary. It may be the case that the 45 minutes allotted to the 2Au to bind to the primary-labeled integrins is too short to allow for a significant amount of binding; a longer binding period could be used to increase the scattering increase from the 1--2Au preparation. However, this would not explain the large amount of nonspecific binding observed in the 2Au sample; additional investigation of that phenomenon is clearly necessary.
